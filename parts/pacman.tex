\section{Gestionnaire de paquets}
\frame{
    \center \LARGE{Gestionnaire de paquets}
}
\begin{frame}
    \frametitle{Qu'est-ce qu'un gestionnaire de paquets ?}
    \begin{itemize}
        \item<1 -> C'est un programme qui gère les tous les paquets du système.
        \item<2 -> Il sert à maintenir votre système à jour, à installer et à
        supprimer des paquets.
        \item<3 -> Il gère aussi le système de dépendances des différents
        paquets.
        \item<4 -> Il peut aussi parfois vous conseiller d'installer des
        dépendances optionelles lors de l'installation d'un paquet.
    \end{itemize}
\end{frame}
\begin{frame}
    \frametitle{apt est votre meilleur ami}
    \begin{itemize}
        \item Les principales features d'apt sont :
        \begin{itemize}
            \item<1 -> installer un paquet : apt-get install <nom du paquet>
            \item<2 -> supprimer un paquet : apt-get remove <nom du paquet>
            \item<3 -> mettre à jour la base de donnée des paquets : apt-get update
            \item<4 -> mettre à jour les paquets du système : apt-get upgrade
            \item<5 -> rechercher un paquet : apt-cache search <expression>
            \item<6 -> une liste plus complète des features d'apt : man apt-get, man apt-cache
        \end{itemize}
    \end{itemize}
\end{frame}
