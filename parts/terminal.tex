\section{Le Terminal} 
\frame{
    \center \LARGE{Le Terminal}
}
\begin{frame}
        \frametitle{Terminal ?}
	\begin{itemize}
	    \item Deux parties :
	\end{itemize}
\end{frame}

\begin{frame}
	\begin{itemize}
		\item<1 -> L'émulateur de terminal
			\begin{itemize}
			    \item urxvt, xterm, gnometerm, screen, tmux, etc.
                \item ce qu'on appelle couramment ``Terminal''
                \item il encadre et lance le shell
			\end{itemize}
	    \item<2 -> Le Shell
            \begin{itemize}
                \item csh, ksh, bash, dash, zsh, etc.
                \item programme qui interprête les commandes
                \item constitue un véritable langage de programmation
	    \end{itemize}
	\end{itemize}
\end{frame}
\begin{frame}
    \frametitle{Commandes ?}
    \begin{itemize}
        \item il est possible d'interagir avec votre OS via des commandes
            \begin{itemize}
                \item<1 -> La plupart du temps CLI = GUI
                \item<2 -> La CLI est légère
                \item<3 -> La CLI est efficace
                \item<4 -> La CLI sent bon
                \item<5 -> La CLI est douce au toucher
                \item<6 -> La CLI a des poils sur le torse
                \item<7 -> La CLI est Bien. Mangez-en.
            \end{itemize}
    \end{itemize}
\end{frame}
