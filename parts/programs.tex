\section{Programmes}
\frame{
    \center \LARGE{Programmes ?}
}
\begin{frame}
    \frametitle{Builtins}
    \begin{itemize}
	\item Les builtins font partie du shell
	\item cat, ls, cp, cp, mv, mkdir, rm
	\item Leur nom n'est pas choisi au hasard
	\begin{itemize}
        \item cat pour Conc\textbf{cat}éner
	    \item cp pour \textbf{C}o\textbf{p}ie
	    \item etc.
	\end{itemize}
    \end{itemize}
\end{frame}

\begin{frame}
        Demo.
\end{frame}

\begin{frame}
    \frametitle{Programmes}
    \begin{itemize}
	\item Votre OS est livré avec les programmes nécéssaires à son fonctionnement
    \begin{itemize}
        \item Certaines distros sont livrées avec énormément de programmes pour
                l'utilisateur (Ubuntu, Mint, etc.)
        \item D'autres sont livrées avec le minimum (Arch, Gentoo, Crux, etc.)
    \end{itemize}
	\item Il est possible d'en ajouter d'autres via un Gestionnaire de Paquets
    \end{itemize}
\end{frame}

\begin{frame}[fragile]
    \frametitle{Combinons des programmes}
    \begin{itemize}
	\item Redirections :
	\begin{itemize}
	    \item >
        \item \begin{verbatim}>>\end{verbatim}
	\end{itemize}
	\item Pipe : `|`
	\begin{itemize}
	    \item ls -la | grep . | grep vim
	\end{itemize}
    \end{itemize}
\end{frame}

\begin{frame}
    \frametitle{Droits}
    \begin{itemize}
        \item lecture (r)
        \item écriture (w)
        \item exécution (x)
	\item Ils sont modifiables
	\begin{itemize}
	    \item chown chgrp chmod
	\end{itemize}
    \end{itemize}
\end{frame}

\frame{
    \center \LARGE{Can I haz root ?}
}

\begin{frame}
    \frametitle{root}
    \begin{itemize}
	\item "root" est l'utilisateur par défaut de votre OS
	\item sudo
	\item su -c
    \end{itemize}
\end{frame}
